%% This is the Chapter 4

%\begin{hyphenrules}{nohyphenation}
\chapter{MODELLING TRAIT-DEPENDENT RATES OF TRAIT EVOLUTION USING CONTINUOUS MATHEMATICAL FUNCTIONS}
%\end{hyphenrules}

\section{Abstract}

To be done.

\section{Introduction}

To be done.

\section{Methods}

\subsection{Description of the model}

Here we use continuous mathematical functions to model the rates of evolution of a trait (i.e., the response trait) under a Brownian-motion model (BM) with respect to the values of another trait (i.e., the predictor trait). The model aims to study the association between the macro-evolutionary patterns of a trait evolving on a phylogenetic tree and the rates of evolution of a second trait on the same tree. Figure \ref{fig:model_example} shows the fundamental concepts of the model. % A summary figure to explain the concept of the model. Try to use colors to associate the simmap on the branches of the tree and the function.
Both the predictor and response traits are continuous traits for the same set of species. The mathematical function represented by the line on Figure \ref{fig:model_example} describes the variation of evolutionary rates for the response trait in function of the trait values of the predictor trait. The difference between the present model and other phylogenetic comparative models of trait evolution with varying rates, such as AUTEUR, bayOU, and BAMM traits, is that these methods only fit rates of evolution for a single trait, informed by the distribution of trait values across the species, and the branch lengths and topology of the phylogenetic tree. In contrast, here we introduce the use of a second trait to inform the variation of rates of evolution across the branches of the tree.

The model can be sub-divided into two components; ancestral values for the predictor trait are mapped to the branches of the phylogenetic tree and evolutionary rate regimes for the response trait are assigned to the phylogeny in function of the map of ancestral predictor trait values. For this we first assign the predictor trait values to $\mathit{k}$ ordered categories defined by $\mathit{k}$ - 1 equidistant breakpoints across the range of the predictor trait values (see x-axis of Figure \ref{fig:model_example}). Then we estimate the evolutionary rate transitions among the $\mathit{k}$ categories with a model that restricts transition rates to happen only between neighbouring states. For example, a transition from the trait value category $\mathit{k_{i}}$ to a larger trait value category $\mathit{k_{i+3}}$ need to be preceded by transitions to and from the intermediary categories $\mathit{k_{i+1}}$ and $\mathit{k_{i+2}}$. The number of categories reflects how fine-grained is the model with respect to the macro-evolutionary patterns of the (continuously distributed) predictor trait; larger $\mathit{k}$ produce a more fine detailed model whereas smaller $\mathit{k}$ yield a more coarse model.
% The comment made in this paragraph calls for a sensitivity test regarding the number of categories used to describe the predictor trait. This is an essential result and should be included in the chapter.

The transition rates between the $\mathit{k}$ categories of the predictor trait can be estimated using a meristic Markov model (Mkn). Here the predictor trait is assumed to evolve under a single homogeneous rate or multiple rate regimes. Homogeneous rate can be described by constraining all transition rates to be equal whereas more complex evolutionary patterns can be described by allowing transition rates to and from each category to vary. Of course, one compare and choose the model that best describes the evolutionary history of the predictor trait across the branches of the phylogenetic tree. As the number of categories ($\mathit{k}$) increases the single rate Mkn model becomes equivalent to a single rate Brownian-motion (BM) model whereas the unconstrained Mkn model spans heterogeneous rate trait evolution models such as multiple rate BM, BM with a trend and Ornstein–Uhlenbeck (OU). However, the Mkn model fits a single transition matrix to the phylogenetic tree and, as a result, can only be equivalent to time-dependent models, such as the accelerated/decelerated (ACDC) and early burst (EB), if the predictor map shows a time-dependent structure.
% Although the free model spans the 'BM with a trend', because of the way the estimation works and because the categories are defined using the tip data, it is very unlikely (maybe impossible) for the 'BM with a trend' model to be one of the estimated models.

In order to map ancestral values of the predictor trait to the branches of the phylogenetic tree we generate multiple stochastic histories for the trait categories using the estimated Mkn transition matrix. Each of these maps associate the branches of the phylogeny with a category for the ancestral values of the predictor trait. Then we use a mathematical function to map these predictor trait regimes to evolutionary rate regimes for the response trait and compute the likelihood of a multiple rates Brownian-motion model with the response trait as the tip data.

\subsection{Mathematical functions and model choice}
% Talk a little that we are using MLE estimation, but that MCMC or even RJMCMC can be implemented. This is more suitable to the Discussion section.

Virtually any mathematical function can be used to map values of the predictor trait categories to evolutionary rate regimes of the response trait. Different functions can be fit to the data using maximum likelihood (ML) and compared using standard model choice approaches such as Likelihood-ratio tests (LRT) for nested or the Akaike information criterion (AIC) for non-nested models. Using model choice criteria that penalizes for the number of parameters (such as AIC) is desirable, since distinct functions of varying complexity can produce identical maps. Some mathematical functions are commonly applied across a series of biological disciplines and are likely to be erected as \textit{a priori} hypotheses for the macro-evolutionary association between a plethora of traits. Figure \ref{fig:bio_functions} shows a collection of functions that describe patterns commonly observed in biological data, especially in studies of trait evolution using phylogenetic trees.

One of the advantages of our approach is that the number of parameters varies with respect to the chosen mathematical function rather than the number of categories used to describe the predictor trait (see Figure \ref{bio_functions}). This is a result of maximizing the likelihood of the data with respect to the parameters of the mathematical function rather than to the rate regimes directly. Thus, we can use a large number of rate regimes in order to model a trait with  (semi-)continuously varying rates of evolution across the phylogeny without increasing the number of parameters of the model. This aspect of the model is similar to the strategy implemented in BAMM traits, which uses an exponential function to describe the continuous decrease or increase of rates of trait evolution through time. % DOUBLE CHECK IF BAMM WORKS LIKE THIS!!

\subsection{Model implementation}

We implemented the model as a R package named \texttt{`phylofx'}. The package offers a simple interface to fit continuous mathematical functions to model the variation of evolutionary rates of a response trait in function of a predictor trait across the branches of a phylogenetic tree. All mathematical functions showed on Figure \ref{fig:bio_functions} are available in the package and can be chosen from a simple menu. The package also have options for users to define their own mathematical functions.

\subsection{Performance simulations}

To check the performance of the method we will focus in three very common, but distinct, nested models: a constant relationship, with homogeneous evolutionary rates ($\sigma^{2}$=0.5); a step function, with two distinct rates separated by an instantaneous transition step ($\sigma^{2}_{left}$=1, $\sigma^{2}_{right}$=0.5, break point=mean of predictor trait); and a linear function, with a continuous relationship between the predictor and the response traits. For the linear function we defined $\beta_{1}$=0.5, set of predictor trait values at the tip as $X$, and defined the intercept such that:

\begin{equation}
\beta_{0} = \beta_{1} \ min X - 0.1
\end{equation}

We generated a phylogenetic tree with 200 species using a pure-birth model (tree height=1). Then we simulated a predictor trait following a single rate BM model ($\sigma^{2}$=0.5) and divided it into 10 trait categories mapped to the branches of the tree. We will refer to the result of this simulation as the true mapped tree. We used the true mapped tree to assign evolutionary rate regimes to the branches following one of the mathematical functions described above and simulated the response trait under a multi-rate BM model. We repeated this process in order to produce 100 datasets for each mathematical function.

In order to fit the models to the generated data, we estimated a meristic Mkn transition matrix for the predictor trait with $\mathit{k}$=5 and equal transition rates. We produced 10 stochastic mapping histories based on this transition matrix and performed a maximum likelihood estimate for each of the mathematical functions under each of the stochastic mapping histories. We chose the best model by comparing the mean pairwise AIC values across all stochastic maps. We repeated model fit and model test for each of the 300 simulated datasets. We computed error rates as the frequency in which the model used to generate the data was not selected as the best model.

Preliminary tests showed that it is often difficult to find the global maximum likelihood for the parameters of the model using minimization algorithms. Thus, we applied three distinct strategies to generate the starting point for the searches. First we generated starting points by drawing from a flat distribution with large range of parameter values (from -400 to 400). Starting points generated using this approach are unlikely to be close to the global maximum but provide an acceptable representation of the parameter space. Second we used a more informed approach by optimizing the parameters of the mathematical functions, prior to the analysis, to produce evolutionary rates equal to $\sigma^{2}$ estimated for a homogeneous BM model with the response trait as the tip data for the tree. Then we defined a narrow range of parameter values (-10 and +10 units around best estimate) to draw starting points from. Finally, we applied the most informative strategy by setting the starting point of the ML searches as the true parameter value for the models. When performing analysis with a model that did not generated the data we set the mathematical function as to minimize the distance relative to the evolutionary rates predicted by the true model. We compare and discuss results among the different search strategies.

\subsection{Sensitivity to trait categorization}
% In a subsequent test I might be able to make a more comprehensible thing by increasing the number of categories of the true model or even making a full continuous simulation.

Our model associates the evolutionary rates of a continuous response trait to the trait values of a continuous predictor trait by transforming the data into ordered categories. On one hand, if the number of trait categories is large enough the trait evolution model converges with continuous models, such as Brownian-motion. On the other hand, this raises concerns about the adequacy of the model and parameter estimates when one applies only a small number of trait categories.

In order to test the sensitivity of the model to the number of trait categories ($\mathit{k}$), we simulated data following the same approach described above for performance simulations. However, we increased the number of species to 500, $\mathit{k}$ to 30 and simulated datasets only with the linear function. We estimated parameter values for the linear model and performed model test using $\mathit{k}$ equal to 30, 15, and 5. Then we computed the distance between estimated parameter values to the true parameter values and the frequency that the generating model was chosen as the best model. 








