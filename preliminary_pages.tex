%Sets non-header pages to same format (location) as header pages, e.g upper-right.
\pagestyle{myheadings}

%Clears pg# from displaying on titlepage
\thispagestyle{empty}

%Titlepage
\begin{center}
Challenges and advances for phylogenetic comparative models of trait evolution\\
\vspace{48pt}
A Dissertation\\
Presented in Partial Fulfilment of the Requirements for the\\
Degree of Doctorate of Philosophy\\
with a\\
Major in Bioinformatics and Computational Biology\\
in the\\
College of Graduate Studies\\
University of Idaho\\
by\\
Daniel Caetano da Silva\\
\vspace{60pt}
Major Professor: Luke Harmon, Ph.D.\\
Committee Members: David Tank, Ph.D.; Jack Sullivan, Ph.D.; Paul Hohenlohe, Ph.D.\\
Department Administrator: Eva Top, Ph.D.\\
\vspace{80pt}
August 2017\\
\end{center}
\pagebreak

%Authorization to Submit Thesis
\addcontentsline{toc}{chapter}{Authorization to Submit Thesis} % Line to add the entry to the table of contents.
\section*{\large{Authorization to Submit Thesis}}
\begin{flushleft}
This dissertation of Daniel Caetano da Silva, submitted for the degree of Doctorate of Philosophy with a major in Bioinformatics and Computational Biology and titled ``Challenges and advances for phylogenetic comparative models of trait evolution," has been reviewed in final form. Permission, as indicated by the signatures and dates given below, is now granted to submit final copies to the College of Graduate Studies for approval.
\end{flushleft}
\begin{singlespace}
\ \ \ \ \ Major Professor:\indent\underline{\makebox[2.8in][l]{\ }}Date\underline{\makebox[1.2in][l]{\ }}\\
\ \ \indent\indent\indent\indent\indent\indent\indent Luke Harmon, Ph.D.\\
\ \\
\ \ \ \indent Committee\\
\ \ \ \indent Members:\indent\indent\ \ \ \ \ \underline{\makebox[2.8in][l]{\ }}Date\underline{\makebox[1.2in][l]{\ }}\\
\ \ \indent\indent\indent\indent\indent\indent\indent David Tank, Ph.D.\\
\ \\
\ \ \indent\indent\indent\indent\indent\indent\ \ \ \ \underline{\makebox[2.8in][l]{\ }}Date\underline{\makebox[1.2in][l]{\ }}\\
\ \ \indent\indent\indent\indent\indent\indent\indent Jack Sullivan, Ph.D.\\
\ \\
\ \ \indent\indent\indent\indent\indent\indent\ \ \ \ \underline{\makebox[2.8in][l]{\ }}Date\underline{\makebox[1.2in][l]{\ }}\\
\ \ \indent\indent\indent\indent\indent\indent\indent Paul Hohenlohe, Ph.D.\\
\ \\
\ \ \indent Department\\
\ \ \indent Administrator:\ \ \ \ \ \ \ \underline{\makebox[2.8in][l]{\ }}Date\underline{\makebox[1.2in][l]{\ }}\\
\ \ \indent\indent\indent\indent\indent\indent\indent Eva Top, Ph.D.\\

\end{singlespace}
\pagebreak

%Abstract
\addcontentsline{toc}{chapter}{Abstract}
\section*{\large{Abstract}}

A phylogenetic tree is a hypothesis of evolutionary relationships among lineages. The branching pattern of such tree tells us the history of groups of species derived from their common ancestors. The length of each branch of this tree represent time, a critical information to study evolutionary processes. Such a tree is the starting point for phylogenetic comparative studies, which aim is to use phylogenies to test hypotheses about macroevolution. On a broad scale, comparative studies can be divided between the study of pattern and processes of lineage diversification and trait evolution. For example, whether the first is concerned with why beetles are one of the most diverse groups among all animals, the latter is intrigued by the beetles' diversity of shapes, sizes, and colors. This dissertation will focus on macroevolutionary patterns and processes underlying phenotypic evolution, especially on identifying challenges and providing advances regarding such analyses.

Here I use and develop statistical models to ask questions such as the association between morphological differentiation and lineage diversification and patterns of evolutionary correlation among several traits. The present dissertation is divided into four chapters. The first tests whether the coloration pattern associated with Neotropical false-coral snakes of the family Dipsadidae has a positive effect on the rates of diversification of the group. For this I collected coloration information for more than 600 species of snakes from a diverse array of sources, including taxonomic descriptions, photos, and direct observations of live and preserved specimens. Then I used the phylogenetic tree of the group to fit a binary state speciation and extiction model (BiSSE) that simultaneously estimates rates of color evolution and associate each color type with diversification rates. After thoughtfully exploring the results due to the known biases associated with this method and applying several additional analysis I concluded that the signal for a dependence between coloration type and diversification of the group is weak. This chapter exemplifies the need for careful exploration of the different factors that might influence any macroevolutionary analysis using phylogenetic trees.

In the second chapter I focused on evolutionary modularity and developed a novel Bayesian approach to study patters of evolutionary correlation among several continuous traits using Markov-chain Monte Carlo. Evolutionary modularity is defined as the pattern in which evolutionary changes of one trait is correlated with evolutionary changes of another trait. The model used in this chapter is based on the evolutionary rate matrix which is a variance-covariance matrix with dimensions equal to the number of traits. The diagonal elements of such matrix have the evolutionary rate for each of the traits whereas the off-diagonals show the pairwise evolutionary covariances. Then, one can use the pairwise covariances in order to study the pattern of evolutionary modularity among continuous traits. In this chapter I also contrast results based on a point estimate of the evolutionary rate matrix using maximum likelihood and the posterior distribution of Bayesian analyses. I show that the lack of uncertainty around parameter estimates under maximum likelihood can bias results since likelihood ratio tests do not take into account the variance around the parameter estimates for the model.

The third chapter is composed by the implementation of the new method described on the second chapter in the widely used programming language and statistical environment R. The package is named \texttt{ratematrix} and it presents a novel approach to deal with the proposal of new variance-covariance matrices for Markov-chain Monte Carlo analyses. The approach separates the variance-covariance matrix into a correlation matrix and a vector of variances. This separation make it possible to explore the parameter space in a more efficient manner while also preventing the proposal of matrices that do not follow the mathematical rules which define a variance-covariance matrix. In this same chapter, I also developed and implemented an extension of Felsenstein's pruning algorithm applied when multiple independent multivariate Brownian-motion rate regimes are fitted to the same phylogenetic tree. This advance of the classic pruning algorithm makes it possible to compute the likelihood of models including a large number of species and traits, since previous implementations suffered from the strong limitations associated with inverting very large matrices.

The fourth chapter introduces the use of mathematical functions to describe the heterogeneity in the rate of evolution of one continuous trait across the branches of a phylogeny as predicted by another trait. Here I developed a new phylogenetic comparative method that utilizes predictive mathematical functions to describe a gradient of rates of evolution for a continuous response trait based on the trait values of a predictor continuous trait mapped to the phylogenetic tree. This approach make it possible to test whether the tempo of macroevolution of a trait is influenced by the gradient of another trait. I also conducted a series of simulations to show that the method can successfully recover the parameter values that generated the data and shows good performance.

In summary, in this dissertation I visit different topics associated with phylogenetic comparative models of trait evolution. Each chapter focus on a current challenge in phylogenetic comparative studies; the association of trait with rates of diversification on chapter one, simultaneous study of several traits on chapters two and three and the correlation between rates of evolution and a potential predictor trait on chapter four. Throughout this dissertation I introduce many advances to the field, especially with respect to the implementation of new models, algorithms, software and the critical evaluation of current practices.

\pagebreak

%Acknowledgements
\addcontentsline{toc}{chapter}{Acknowledgements}
\section*{\large{Acknowledgements}}

\ \ \ \ \ It is not easy to forget the process of moving from Sao Paulo (Brazil) to Moscow (USA). Such a decision represented for me the best move I could ever make in my career. For the American reader I want to ask you to look around and imagine that your profession of choice is conducted in another language. Now also try to think that the next big conference in your field will be overseas, the next also, and the other too. You now also have to worry if a reviewer will ask whether a native speaker reviewed your writing before sending your manuscript to review. That is reality on the majority of countries around the world that do science and everyday concerns for both students and faculty. So, yes, coming to US to pursue my doctorate degree was a big thing.

Fortunately, all the anxiety that I felt after realizing the size of Moscow, ID, went away when I started working with the fantastic community of evolutionary biologists at the University of Idaho. The frequent interactions with students and faculty makes this place warm to the heart, more than a necessity given the extreme low temperatures during the dark afternoons of winter. The constant open conversation and clear focus on scientific arguments rather than academic hierarchies makes any student feel as part of something bigger. This environment makes us want to grow and to develop as far as we can. The whole department should be proud of the energy that it transmits.

This is also a place of routine, but in a good way. The schedule of the week is quite clear: work, PEES, IBEST Lunch, meet the speaker, and, of course, the Friday: PURGE, check email, Biology seminar, and... is already 5PM, so beers. I am guilty of complaints about so many things to do sometimes. However, I am sure that I am going to miss all this so much! Among all the things, PURGE will have the longest mark on me, both as a person and as a professional. The opportunity of discussing published articles with fellow students and professors weekly is just fantastic! I learned so much with everyone. Learned that it is good to know and even better to know you don't know. Learned to listen and make yourself listened. And, of course, the occasional practice of talking about a manuscript you failed to read will, I am sure, be handy some day.

The colleagues, collaborators and friends that made me company on this five years journey also deserve many thanks. Professors Luke Harmon, Dave Tank and Jack Sullivan were always there to discuss any topic and for this I am hugely grateful. The Harmon lab, my home, is a group of big ideas and bold projects. I learned a lot from all lab-mates. Thanks to Matt Pennell for helping me with all the first steps as a graduate student and to inspire (also provoke) me to pursue big accomplishments. Thanks Denim Jochimsen for helping me with all the snake-things. Many thanks to Rafael Maia and Eliot Miller, was super cool to have you in the lab and bring much more diversity of topics to the table. Special thanks to Rosana Zenil-Ferguson for raising the level of the statistical work of everyone in the lab and for the patience to explain all the things. Another super special thanks to Josef Uyeda. I am happy to say that Josef is a model of the scientist I hope one day to be and I am super proud of having shared a lot of ideas and beers with you. Finally, my big thanks to Luke Harmon! I am still impressed with how luck I was to have the opportunity to join your team. Being your student was an amazing experience and a very important turning point in my academic career. I will aim high and do all the things Luke, I promise!

Many thanks to the Tank lab, my second home, and all the plant people that effectively help the Harmon lab to keep things real. Since Dave Tank moved to the Biology Department it has been great to have him around. You helped me a lot Dave, thanks! Thanks to Hannah Marx and Simon Uribe-Convers, both PhDs for some time already. Many thanks also to Diego Morales-Briones and Sarah Jacobs for making me company in the office, discussing all the science and also having all the fun. Many thanks also to Megan Ruffley, Ian Gilman and Sebastian Mortimer, you are all great and wish all the good things for your next steps!

And, of course, many thanks to the Moscow crew! Too many names to list but, hey, you know all who you are. See you at BBBBBBBBBB!!

\pagebreak

%Dedication
\addcontentsline{toc}{chapter}{Dedication}
\vspace*{\fill} % This centers the text vertically
\begin{center} % This centers the text horizontally

\begin{large}
\textbf{Dedication} \\
\end{large}
\indent To my parents for all the effort in providing me more that they ever had. \\
\indent All that I got and will get is because of their support, patience and love. \\
\indent Para meus pais por todo o esforço para me dar muito mais do que eles tiveram na vida. \\
\indent Tudo o que consegui até hoje foi por causa do seu apoio, paciência e amor.

\end{center}
\vspace*{\fill}
\pagebreak

%Table of Contents
\addcontentsline{toc}{chapter}{Table of Contents}
\tableofcontents
\pagebreak

%List of Tables
\addcontentsline{toc}{chapter}{List of Tables}
\listoftables
\pagebreak

%List of Figures
\addcontentsline{toc}{chapter}{List of Figures}
\listoffigures
\pagebreak